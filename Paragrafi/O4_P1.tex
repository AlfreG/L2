\subsection{Ottica geometrica}
    \subsubsection{Legge di Cartesio}
        Inserire disegno schema esperimento\\
        Emettitore e ricevitore sono stati posizionati al centro di snodo delle guide. In modo che il segnale senza riflessione fosse a circa metà del misuratore, senza amplificazione (x1), lettura 0.32 $\pm$ 0.02 mA .
        %
        \begin{table}[H]
    \begin{center}
    \begin{tabular}{|c|c|}
        \hline
            $\theta_i$  & $\theta_r$    \\
            deg         & deg           \\
            $\pm$ 2     & $\pm 1$         \\ \hline
%    
            40 & 80  \\
            30 & 62  \\
            50 & 100 \\
            60 & 118 \\
            70 & 137 \\ \hline
%  
    \end{tabular}
    \end{center}
    \caption{ Angoli di incidenza, riflessione e intensità. Distanza ER dal centro 20 cm.}
    \label{O4_P1_cartesio}
\end{table}

        \paragraph{Note}{ L'emettitore emette onde anche fuori dall'altezza dello specchio (perchè è troppo basso) quindi si perde intensità anche per quello.\\
        Emettitore è lasco sul binario, quindi abbiamo aumentato l'errore.}
%
%
%
    \subsubsection{Rifrazione}
%
        \paragraph{ Indice di rifrazione del polistirolo}
        Il prisma di polistirolo, posto lungo l'asse ottico non ha provocato variazioni sensibili dell'intensità del segnale. La verifica è stata ottenuta togliendo e inserendo il prisma a diversi angoli di inclinazione del ricevitore, verificando l'assenza di variazioni significative del segnale ricevuto. Si pone perciò
%
            \[  n_{pol} = n_{aria}.  \]
%
        \paragraph{ Indice di rifrazione dello styrene }
        La misura dell'angolo del prisma $ \alpha = 23 \pm 1 $ gradi, è stata ottenuta con l'utilizzo del goniometro e del profilo sagomato del prisma su un foglio di carta.\\
%
        Il prisma è stato posizionato con la faccia verso il raggio incidente ortogonale all'asse ottico, in modo da creare una sola rifrazione.\\
%
        L'angolo $ \theta_{r} $ è l'angolo di rifrazione del raggio uscente dal prisma, misurato con il goniometro imperniato sulle guide metalliche.\\
            $$ \alpha = 23 \pm 1 \mathrm{°}$$ 
            $$ \theta_r =  33 \pm 1 ° $$
%        
        Posto $ n_{aria} = 1 $, il rapporto fra i seni fornisce l'indice di rifrazione del materiale
    
            $$ n = \frac{\sin( \alpha )}{ \sin( \theta_{r} ) } = 1.39 \pm 0.04. $$ 
