Laser utilizzato: Red Helium-Neon, $\lambda = 633$ nm.

\subsection{Interferometro di Fabry-Perot}

Si è misurato lo spostamento dello specchio mobile ($d$) nel conteggio di $\Delta k = 50 $ frange. Si è ricavata la lunghezza d'onda con:
$$\lambda = \frac{ 2 d \cos \theta }{\Delta k} $$
con l'approssimazione $\theta \approx 0$ (angolo tra l'asse ottico ed il riferimento sullo schermo per il conteggio delle frange).\\\\
%
Si sono ripetute le misure partendo da punti diversi della scala del micrometro, in modo da ottenere una curva di calibrazione del micrometro, ma non si sono osservati grossi discostamenti dal valore atteso di lughezza d'onda.\\\\
%
Tabella misurazioni:
%
\begin{table}[H]
    \centering
    \begin{tabular}{|c|c|c|}
    \hline
        $d_0$ iniz.	&	$d$	&	$\lambda$ \\
        $\mu m$	&	$\mu m$	&	$nm$ \\
        ($\pm 1$)   & ($\pm 1$) & ($\pm 42$)\\
    \hline
        800	&	17	&	680	\\
        700	&	16	&	640	\\
        600	&	17	&	680	\\
        400	&	16	&	640	\\
        300	&	16	&	640	\\
        200	&	16	&	640	\\
        100	&	16	&	640	\\
        6	&	16	&	640	\\
    \hline    
    \end{tabular}
\end{table}

%
L'errore di $\pm 42$ nm su $\lambda$ è calcolato mediante propagazione degli errori.\\
Il risultato finale ottenuto come media ed errore della media è:
$$\lambda = 650 \pm 16\; nm $$
Il valore vero di $\lambda$ si trova entro l'errore del risultato.
%

