\subsection{Conclusioni}
\label{sec:C3_final}

\paragraph{Onda quadra}{ Nella pratica risulta impossibile realizzare un'onda perfettamente quadra poichè quando la tensione viene variata bruscamente il circuito ha dei tempi di risposta non istantanei. Se questo transiente dura $\delta t$, non nullo, mentre la parte a potenziale costante dura $\Delta t$, si ha che all'aumentare della frequenza  $\Delta t$ diminuisce. L'attesa è che $\delta t$ sia molto più influente ad alte frequenze poichè è maggiore il rapporto $\delta t/\Delta t$. Questo fenomeno non risulta evidente dai dati sperimentali.}
%
%
%
\paragraph{Metodo alternativo per la stima del tempo caratteristico}{Si osserva che all'aumentare della frequenza l'ampiezza massima della tensione aumenta in maniera monotona fino a raggiungere un massimo. Detto $\tau$ il tempo caratteristico di carica del condensatore e $T = 1/\nu$ il periodo dell'oscillazione del generatore, noto che $Q=V/C$ si osserva che:
\begin{itemize}
    \item per $T$ molto minore di $n\tau$ il condensatore non riesce a raggiungere la carica completa, quindi la tensione risulta bassa.
    \item per $T$ circa dell'ordine di $n\tau$ l'ampiezza raggiunge un massimo poichè la carica avviene in maniera completa
    \item per $T$ molto maggiore di $n\tau$ l'ampiezza si satura e il profilo tende a un'onda quadra, per l'argomento del paragrafo precedente.
\end{itemize}
Detta $\nu^*$ la frequenza in corrispondenza al massimo di ampiezza e $T^*$ il relativo periodo, se si assume $T^*$  è circa pari a $n\tau$, allora, noto R, si può ricavare C da:
    $$\tau =  CR \Rightarrow C \approx T^*/nR $$ }
%
\paragraph{Confronto misure} In questo esperimento le misure su induttanze e capacità sono state condotte nel dominio della frequenza (parte 1, impedenza e funzione di trasferimento) e nel dominio del tempo (parte 2, tempi caratteristici di carica e scarica). \\
%
Un confronto diretto tra le due misure è possibile solo per la capacità da $11$ $nF$, dove si evidenzia un parziale accordo fra le due.\\
%
Dai dati riportati, non risulta chiaro se uno dei due metodi sia preferibile all'altro quanto a precisione, a causa della forte diversità degli errori relativi da componente a componente.\\
%
A questo proposito si osserva che, nelle misure in corrente alternata, a causa della forte differenza tra errori sperimentali ed effettiva varianza dei residui di regressione, gli errori delle variabili dipendenti sono stati ricalcolati in modo da normalizzare il chi quadrato,
%
\begin{align*}
\chi^{2} &= \sum_{i=1}^{n}\frac{ ( y_{i}^{o} - y_{i}^{s})^{2} }{\sigma_{i}^{2}}    \\
\sigma^{2}_{ric} & = \sum_{i=1}^{n}\frac{ ( y_{i}^{o} - y_{i}^{s})^{2} }{gdl} 
\end{align*}
%
Senza tale normalizzazione gli errori relativi delle misure in corrente alternata risulterebbero in media più alti di quelli riportati, ma perchè sovrastimati rispetto all'effettiva varianza dei residui.
%
\begin{table}[H]
\begin{center}
\begin{tabular}{|c|c|c|c|}
\hline

 
Q.tà & U.tà  & \multicolumn{2}{|c|}{Misure}  \\
\multicolumn{2}{|c|}{}  
& Corr. Alternata & Corr. Impulsata \\\hline

\multirow{2}{*}{$L_{1}$} 
& \multirow{6}{*}{$[H]$}
& 0.107        &   \\
&
& $\pm 0.9\%$  &   \\

\multirow{2}{*}{$L_{2}$}
& 
&  0.0451        &   \\
&
&  $\pm 4\%$  &   \\

\multirow{2}{*}{$L_{3}$} 
&
&         &  0.038 \\
&
&         & $\pm 2.6\%$   \\\hline


\multirow{2}{*}{$C_{1}$} 
&  \multirow{4}{*}{$[nF]$}
& 11.38       & 11.7  \\
& 
& $\pm 1\%$  &  $\pm 3.4\%$   \\

\multirow{2}{*}{$C_{2}$} 
&  
& 47  &   \\
&
& $\pm 4\%$  &   \\\hline

\hline
\end{tabular}
\end{center}

\caption{Confronto precisone stime di capacità e induttanza in corrente alternata e impulsata. Errori relativi percentuali sono riportati sotto le relative stime.}
\label{C3_finale}

\end{table}

