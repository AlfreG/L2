\subsection{Teoria: misure di impedenza in circuiti RC e RL in corrente alternata}

Consideriamo un circuito nel quale tensione e corrente variano nel tempo secondo una legge di tipo sinusoidale $V(t)=V_0 \cos(\omega t)$ ed $I(t)=I_0 \cos(\omega t)$. Consideriamo elementi circuitali resistivi (R), capacitivi (C) e induttivi (L) ed andiamo ad analizzare l'equazione caratteristica di ciascun componente per determinare l'espressione della differenza di potenziale ai capi di esso.\\\\
%
\textbf{Resistenze}:
L'equazione caratteristica di un resistore attraversato dalla corrente $I_0\cos(\omega t)$ è: 
    $$V_R (t) = RI(t) = RI_0 \cos(\omega t)$$
si nota quindi che ai capi di un resistore, la differenza di potenziale e la corrente sono in fase, e che tra le ampiezze sussiste la relazione:  
    $$V_0 = RI_0$$
Si osserva inoltre che non si ha alcuna dipendenza dalla pulsazione $\omega$\\\\
%
\textbf{Capacità}:
La capacità è definita come $C = Q/V_C$, dove $Q$ è la carica accumulata sul condensatore e $V_C$ è la tensione ai capi di esso. L'equazione del circuito, se viene applicata una tensione sinusoidale $V_C(t)=V_0 \cos(\omega t)$, diventa:
    $$Q(t) = CV_C(t) = CV_0 \cos(\omega t) $$
%    
La corrente quindi $I(t)=\mathrm{d}Q(t)/\mathrm{d}t$:
    $$I(t) = C \frac{\mathrm{d}V(t)}{\mathrm{d}t} = \omega C V_0 (-\sin \omega t) = \omega C V_0 \cos \left(\omega t + \frac{\pi}{2}\right) $$
%
La tensione ai capi di un condensatore in un circuito in cui scorre la corrente $I(t)=I_0 \cos(\omega t)$ si trova risolvendo l'equazione $I(t)=C\mathrm{d}V_L(t)/\mathrm{d}t$, ove si è posta la costante di integrazione (il potenziale $V_T$) a 0:
    $$V_C = \frac{1}{C} \!\cdot{}\!\!\!\! \int \!I(t) \mathrm{d}t = \frac{1}{\omega C} \!\cdot{}\! \sin(\omega t)  + V_T = \frac{1}{\omega C} \!\cdot{}\! \cos \left(\omega t - \frac{\pi}{2}\right)$$
%
La tensione risulta quindi sfasata di $-\pi/2$ rispetto alla corrente e il suo valore in modulo è:
    $$V_0 = \frac{1}{\omega C} I_0$$
%    
La quantità $1/\omega C$ si definisce \textit{reattanza} del condensatore.\\\\
%
\textbf{Induttanze}:
Per un elemento induttivo di resistenza trascurabile, ai capi del quale viene applicata una tensione alternata $V_L(t) = V_0 \cos(\omega t)$,  l'equazione del circuito è:

    $$ V_L(t) = L \frac{\mathrm{d}I(t)}{\mathrm{d}t} \Rightarrow \frac{\mathrm{d}I(t)}{\mathrm{d}t} = \frac{V_0 \cos(\omega t)}{L} \Rightarrow$$
    
    $$\Rightarrow I(t) = \frac{V_0}{\omega L}\sin(\omega t) = \frac{V_0}{\omega L}\cos\left(\omega t - \frac{\pi}{2}\right) $$

Dunque, la tensione ai capi di un induttore in cui scorre una corrente $I(t)= I_0\cos(\omega t)$ è:
    $$V_L =L\frac{\mathrm{d}I(t)}{\mathrm{d}t} =\omega L I_0 \cos\left(\omega t + \frac{\pi}{2}\right) $$
e in modulo:
$$V_0=\omega L I_0$$    
Tensione e corrente risultano sfasate di $+\frac{\pi}{2}$ e la reattanza dell'induttore è $\omega L$\\
%
\subsubsection{Simbolismo complesso}
Se per tensione e corrente utilizziamo la rappresentazione complessa $V(t)=V_0 e^{j(\omega t + \phi)}$ e $I(t)=I_0 e^{j\omega t}$, allora Tensione e corrente (complesse) risultano proporzionali.\\
%
Il coefficiente di proporzionalità tra tensione e corrente complesse è detto \textit{impedenza}, un numero complesso che racchiude gli effetti del circuito: il suo modulo indica il rapporto tra le ampiezze di V e I reali e la sua fase indica il loro sfasamento. 
    $$ Z = Z_0 e^{j\phi} $$  
\paragraph{Nota}{
        Indicheremo con $\phi$ lo sfasamento della tensione rispetto alla corrente. Cioè se $I(t)=I_0 \cos(\omega t)$ allora $V(t)=V_0\cos(\omega t + \phi)$}
    \\
Se rappresentato in forma algebrica $Z = R + jX$, la sua parte reale indica la componente resistiva del circuito e la sua parte immaginaria la componente reattiva (reattanza sopra definita è la parte immaginaria dell'impedenza).
    $$Z=R+jX$$
dove $j$ è l'unità immaginaria.\\
%
Il reciproco dell'impedenza è detto \textit{ammettenza}  $Y=1/Z$

Ripetendo i calcoli per i potenziali ai capi di R, C e L con il formalismo complesso, si trova che:
%
\begin{center}
\begin{tabular}{ l l l l l }
 $Z_R$ & $Z = R$ & $R \cdot{} \mathrm{exp}(j0)$ & $Z_0= R C$ & $\phi= 0$\\
 %
$Z_C$ & $Z = 1/j\omega C$ & $1/\omega C \cdot{} \mathrm{exp}(-j\pi/2)$ & $Z_0= 1/\omega C$ & $\phi= -\pi/2$\\  
 %
 $Z_L$ & $Z = j\omega L$ &  $\omega L \cdot{} \mathrm{exp}(j \pi/2)$   & $Z_0=\omega L$ & $\phi= \pi/2$
\end{tabular}
\end{center}
%
In circuiti composti da serie e parallelo di R, L, e C, si sommano impedenze in serie e ammettenze in parallelo con le consuete leggi di Kirchoff.\\
%
\subsubsection{Calcolo impedenze su circuiti RC e RL}
L'impedenza totale del circuito è $Z = \sum_{i=1}^n R_i = R + jX$. Il suo modulo è dato da $|Z| = \sqrt{R^2 + X^2}$ e la sua fase da $\phi = \arctan (X/R)$. Seguono calcoli dell'impedenza per circuito RC ed RL.\\\\
%
\textbf{Serie RC}
    $$Z = Z_R+Z_C = R + \frac{1}{j\omega C} = \sqrt{ R^2+\frac{1}{(\omega C)^2} } \cdot{} e^{j\phi} \rightarrow $$
    $$\rightarrow|Z|= \sqrt{ R^2+\frac{1}{(\omega C)^2} } \quad,\quad \phi = -\arctan \left(\frac{1}{\omega R C}\right)$$
Attendiamo dunque che a basse frequenze il circuito abbia un comportamento capacitivo (quindi che la fase dell'impedenza tenda a $-\pi/2$) e che ad alte frequenze abbia un comportamente resistivo (che la fase dell'impedenza tenda a 0).\\
Questo supporta l'intuizione in quanto per le basse frequenze il rapporto tra la costante di tempo del circuito e il periodo è molto basso. Ciò significa che in un periodo il numero di scariche della capacità (e quindi il flusso di carica nell'intervallo di tempo) è molto basso, e, nel limite di corrente continua, il circuito risulta aperto (non passa corrente attraverso C). La differenza di potenziale ai capi di C corrisponde al massimo di ampiezza della tensione. Come vedremo nella sezione successiva la funzione di trasferimento tende al valore 1.\\
Ad alte frequenze il rapporto tra il tempo caratteristico e il periodo dell'oscillazione è alto e la capacità ha numerose scariche in un periodo, permettendo un transito netto di corrente con una componente resistiva molto bassa. La tendenza è quindi quella di un cortocircuito, e la funzione di trasferimento tende a zero.

Se consideriamo anche la resistenza interna del generatore ($r_g$), nominalmente di $50 \Omega$, nel calcolo precedente è sufficiente sostituire il termine R alla somma $R + r_g$. L'impedenza diventa quindi:
 $$|Z|= \sqrt{ (R+r_g)^2+\frac{1}{(\omega C)^2} } \quad,\quad \phi = -\arctan (\frac{1}{\omega (R+r_g) C})$$
%
Di seguito le resistenze che abbiamo utilizzato e l'influenza della resistenza del generatore.
\begin{center}
    \begin{tabular}{|c c c|}\hline
            &                   & effetto $r_g$  \\
         R1 & 14870 $\Omega$    & 0.3\%      \\
         R2 & 677 $\Omega$      & 7\%   \\\hline
    \end{tabular}
    \label{tab:C3_P1_effetto_rg}
\end{center}
%
Per R2 ci aspettiamo che possa avere infuenza significativa\\

\textbf{Serie RL}
$$ Z = Z_R + Z_L = R + j\omega L = \sqrt{R^2+(\omega L)^2} \cdot{} e^{j\phi} $$
$$ |Z| = \sqrt{R^2+(\omega L)^2} \quad , \quad \phi = \arctan \left( \frac{\omega L}{R}\right) $$
Attendiamo dunque che a basse frequenze il circuito presenti un comportamento resistivo (cioè che la fase dell'impedenza tenda a zero) menre ad alte frequenze un comportamento induttivo (e quindi il tendere della fase a $+\pi/2$).\\
Intuitivamente, quando la frequenza è bassa (cioè si ha una tendenza alla corrente continua) la variazione di flusso del campo magnetico autoindotto avviene molto lentamente. A ciò corrisponde una forza elettromotrice autoindotta molto bassa, per la legge di Faraday $fem = -\mathrm{d}\Phi_B/\mathrm{d}t$, e quindi un effetto induttivo molto debole. Nel limite di corrente continua, l'induttore dovrebbe avere tendenza di corto circuito (se la resistenza interna è molto piccola) o di resistore (se la resistenza interna è apprezzabile). Al contrario, a frequenze elevate, si ha forte variazione di flusso del campo magnetico, intensa forza elettromotrice autoindotta e notabile effetto induttivo.\\
%
Se consideriamo anche la resistenza interna del generatore ($r_g$), nominalmente di $50 \Omega$, e dell'induttanza ($R_L$) di cui abbiamo tratto una prima stima con una misurazione da multimetro pari a $60 \Omega$, nel calcolo precedente è sufficiente sostituire il termine R alla somma $R + r_g + R_L$. L'impedenza diventa quindi:
$$ |Z| = \sqrt{(R + r_g + R_L)^2+(\omega L)^2} \quad , \quad \phi = \arctan \left( \frac{\omega L}{(R + r_g + R_L)}\right) $$
Non sapendo il funzionamento esatto del multimetro nel misurare la resistenza di una induttanza, assumiamo il valore di $60 \Omega$ solo come prima stima al quale attribuiamo solamente l'errore nominale dello strumento di misura, riservandoci di ricavare un valore più preciso dall'adattamento dei dati al modello.\\

Di seguito le resistenze che abbiamo utilizzato e l'influenza della resistenza dell'induttore.
\begin{center}
    \begin{tabular}{|c c c|}\hline
            &                   & effetto $r_L$  \\
         R1 & 14870 $\Omega$    & 0.4\%      \\
         R2 & 677 $\Omega$      & 9\%   \\\hline
    \end{tabular}
    \label{tab:C3_P1_effetto_rL}
\end{center}
Come prima, ci aspettiamo effetti maggiori su R2.\\\\
%
%
\subsubsection{Aspettative}
\begin{center}
    \begin{tabular}{ c c c c c}
        \hline
        Circuito & Basse Frequenze & Alte Frequenze & $\|Z\|$ & $\phi$ \\ \hline
        RC &  Capacitivo & Resistivo & $\sqrt{ R^2+\frac{1}{(\omega C)^2} }$ &$-\arctan (\frac{1}{\omega R C})$\\\\
        RL & Resistivo & Induttivo & $\sqrt{R^2+(\omega L)^2}$  & $\arctan \left( \frac{\omega L}{R}\right)$
    \end{tabular}
    \label{tab:C3_P1_aspettative_RC_RL}
\end{center}
%
\textbf{Influenza Resistenza generatore}: probabile effetto su R2\\
\textbf{Influenza Resistenza induttore}: probabile effetto su R2\\



%%%%%%%%%%%%%%%%%%%%%%%%%%%%%%%%%%%%%%%%%%%%%%%%%%%%%%%%%%%%

%%%%%%%%%%%%%%%%%%%%%%%%%%%%%%%%%%%%%%%%%%%%%%%%%%%%%%%%%%%%

%%%%%%%%%%%%%%%%%%%%%%%%%%%%%%%%%%%%%%%%%%%%%%%%%%%%%%%%%%%%

%%%%%%%%%%%%%%%%%%%%%%%%%%%%%%%%%%%%%%%%%%%%%%%%%%%%%%%%%%%%

\subsection{Teoria: funzione di trasferimento in circuiti RC e RL in corrente alternata}

Dato un qualsiasi circuito, si ha interesse a valutare la relazione tra segnali di output e di input in relazione alla frequenza della corrente considerata. In particolare si vuole una stima quantitativa e univoca di come viene variata la tensione in ingresso da quel particolare circuito.\\
Si definisce allora la \textit{funzione di trasferimento} $H(\omega)$ del circuito come rapporto tra la variazione di tensione tra uscita e l'ingresso e la tensione in ingresso del circuito o di una parte di esso.\\ 
Poichè $V_{Out}$ e $V_{In}$ sono tensioni complesse, anche H è un numero complesso, il cui modulo indica il rapporto tra le ampiezze tra le tensioni e la cui fase lo sfasamento tra le tensioni.
Di seguito per i conti utilizziamo $v$,$i$ per indicare corrente e tensione complesse e $V$,$I$ per quelle reali
    $$ H(\omega) \equiv \frac{v_{Across}}{v_{In}}$$
In quest'esperienza vogliamo valutare la funzione di trasferimento ai capi di un'unica componente del circuito, L o C.\\\\
%
\textbf{Calcolo per RC}: 
Siano $v_{C}(t)$ la tensione attraverso la capacità e $i(t)$ la corrente che scorre attraverso il circuito (omettiamo le dipendenze temporali dei calcoli dove non strettamente necessario). Nella terminologia precedente:
    $$v_{Across}=v_C = Z_C i = \frac{i}{j\omega C} \quad\mathrm{e}\quad v_{In} = Z_{tot}i = (Z_C+Z_R)i = \left(\frac{1}{j\omega C} + R\right)i$$ 
quindi la funzione di trasferimento:
    $$H(\omega) = \frac{v_{Across} }{v_{In} } = \frac{1/j\omega C}{R+1/j\omega C} = \frac{1}{1+j\omega RC}$$
%
$H$ è un numero complesso di modulo e fase:
    $$|H(\omega)| = \frac{1}{\sqrt{1+(\omega RC)^2} } \quad\mathrm{e}\quad Arg(H(\omega)) = \arctan (-\omega RC)$$
    %
Ci attendiamo quindi che: a basse frequenze la funzione di trasferimento abbia modulo tendente a 1 e fase tendente a 0, ad alte frequenze il modulo tende a 0 e la fase a $-\pi/2$.\\\\
%
\textbf{Calcolo per RL}: Siano $v_{L}(t)$ la tensione attraverso l'induttanza e $i(t)$ la corrente che scorre attraverso il circuito (omettiamo le dipendenze temporali dei calcoli dove non strettamente necessario). Nella terminologia precedente:
    $$v_{Across}=v_L = Z_L i = j\omega L \quad\mathrm{e}\quad v_{In} = Z_{tot}i = (Z_L+Z_R)i = (j\omega L + R )i$$ 
quindi la funzione di trasferimento:
    $$H(\omega) = \frac{v_{Across} }{v_{In} } = \frac{j\omega L}{R+j\omega L} = \frac{j\omega L}{R+j\omega L}$$\\
$H$ è un numero complesso di modulo e fase:
    $$|H(\omega)| = \frac{\omega L}{\sqrt{R^2+(\omega L)^2} } \quad\mathrm{e}\quad Arg(H(\omega)) = \arctan \left(\frac{R}{\omega L}\right)$$
    %
Attendiamo quindi che per frequenze basse il modulo tenda a zero e la fase a $+\pi/2$, mentre a frequenze alte il modulo a 1 e la fase a 0.
%
\subsubsection{Aspettative}
\begin{center}
    \begin{tabular}{ |c|c c|c c|}
        \hline
        Circuito& Basse &Frequenze & Alte &Frequenze \\ 
                & $\| H(\omega) \| $ & $Arg(H(\omega))$ & $\|H(\omega)\|$ & $Arg(H(\omega)$ \\\hline
                &           &            &          &  \\
        RC      & Tende a 1 & Tende a 0  & Tende a 0& Tende a $-\pi/2$\\
                &           &            &          &   \\
        RL      & Tende a 0 & Tende a 1  & Tende a $+\pi/2$ & Tende a 0  \\ \hline
    \end{tabular}
    \label{tab:C3_P1_aspettative_RC_RL}
\end{center}
Possiamo osservare quindi che il circuito RC ai capi della capacità si comporta come un filtro che smorza le frequenze alte (\textit{Filtro Passa-Basso}), mentre il circuito RL ai capi dell'induttanza si compoorta come un filtro per le frequenze basse (\textit{Filtro Passa-Alto}).\\

Se invece si invertono la posizione di R e C (o L) e si misura la funzione di trasferimento ai capi della resistenza, i comportamenti di filtro si invertono: RC diventa \textit{Passa-Alto} e RL \textit{Passa-Basso}.\\\\
%
\textbf{Inversione di R e C}: le tensioni sono (secondo la terminologia precedente)
    $$v_{Across} = Z_Ri = Ri\quad\mathrm{e}\quad v_{In} = (Z_R+Z_C)i = (R + 1/j \omega C )i $$
    
funzione di trasferimento:
    $$H(\omega) = \frac{v_{Across} }{v_{In} } = \frac{j\omega CR}{j\omega CR + 1}$$
    
modulo e fase:
    $$ |H(\omega)| = \frac{\omega CR}{\sqrt{(\omega CR)^2 + 1}} \quad\mathrm{e}\quad Arg(H(\omega)) = \arctan (1/\omega Cr) $$
%
Si osservi come per basse frequenze $|H|$ tende a 0 e ad alte frequenze tende a 1 (\textit{Passa-Alto})\\\\
%
%
\textbf{Inversione di R e L}: le tensioni sono (secondo la terminologia precedente) 
    $$v_{Across} = Z_Ri = Ri\quad\mathrm{e}\quad v_{In} = (Z_R+Z_L)i = (R + j \omega L )i $$
funzione di trasferimento:
    $$H(\omega) = \frac{v_{Across} }{v_{In} } = \frac{R}{R + j\omega L}$$
modulo e fase:
    $$ |H(\omega)| = \frac{R}{\sqrt{(\omega L)^2 + R^2}} \quad\mathrm{e}\quad Arg(H(\omega)) = \arctan \left(-\frac{\omega L}{R} \right) $$
%  
Si osservi come per basse frequenze $|H|$ tende a 1 e ad alte frequenze tende a 0 (\textit{Passa-Basso}).\\

