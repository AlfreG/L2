\subsection{Velocità della luce}

Distanza L2 - specchio rotante:	$B = 0.315 \pm 0.001$ m;\\\\
%
%
Distanza specchio rotante - specchio fisso:	$D = 4.980 \pm 0.005$ m;\\
%
Distanza focale L2:	$f = 0.249 \pm 0.001$ m;\\
%
Distanza punto di convergenza del fascio laser (sorgente) - L2: $A = 0.261 \pm 0.001$ m, calcolato da
$$ A = \frac{(B+D)f}{B+D-f} $$
%
La velocità della luce si trova mediante la formula
$$ c = \frac{8 \pi A D^2 (\omega_{sx} + \omega_{dx})}{(D+B) \Delta S} $$
%
dove $\Delta S$ è lo spostamento letto sul micrometro del punto, osservato tra i due diversi versi di rotazione dello specchio.\\\\
%
La misura di $\Delta S$ è stata ripetuta 10 volte per ognuna delle due velocità di rotazione. Ecco di seguito riportati i dati:
\begin{table}[H]
    \centering
    \begin{tabular}{|c|c|c|c c|}
    \hline
    $x_L$ [mm]	&	$x_R$ [mm]	&	$\Delta S$ [mm]	\\
    ($\pm 0.01$)	&	 ($\pm 0.01$)	& ($\pm 0.02$)	\\
    \hline
        12.77	&	13.07	&	0.30	\\
        12.75	&	13.07	&	0.32	\\
        12.76	&	13.05	&	0.29	\\
        12.79	&	13.09	&	0.30	\\
        12.76	&	13.07	&	0.31	\\
        12.77	&	13.08	&	0.31	\\
        12.80	&	13.08	&	0.28	\\
        12.76	&	13.07	&	0.31	\\
        12.76	&	13.09	&	0.33	\\
        12.77	&	13.08	&	0.31	\\
    \hline
    \end{tabular}
    \caption{$\omega = 1550$}
    %\label{tab:my_label}
\end{table}
\begin{table}[H]
    \centering
    \begin{tabular}{|c|c|c|c c|}
    \hline
    $x_L$ [mm]	&	$x_R$ [mm]	&	$\Delta S$ [mm]	\\
    ($\pm 0.01$)	&	 ($\pm 0.01$)	& ($\pm 0.02$)	\\
    \hline
        12.49	&	12.68	&	0.19	\\
        12.53	&	12.68	&	0.15	\\
        12.55	&	12.69	&	0.14	\\
        12.49	&	12.68	&	0.19	\\
        12.50	&	12.69	&	0.19	\\
        12.50	&	12.68	&	0.18	\\
        12.49	&	12.67	&	0.18	\\
        12.50	&	12.68	&	0.18	\\
        12.49	&	12.68	&	0.19	\\
        12.83	&	12.98	&	0.15	\\
    \hline
    \end{tabular}
    \caption{$\omega = 750$}
    %\label{tab:my_label}
\end{table}
%
Valori ottenuti per la velocità della luce (errore per ogni misura calcolato con propagazione degli errori, poi effettuata la media pesata):
\begin{itemize}
    \item con $\omega_{dx} = \omega_{sx} = 1500 $, si trova $ c = 3.00 \cdot 10^8 \pm 0.01 \cdot 10^8 $, in perfetto accordo con il valore vero;
    \item con $\omega_{dx} = \omega_{sx} = 750 $, si trova $ c = 2.57 \cdot 10^8 \pm 0.09 \cdot 10^8 $, che si discosta del 14\% rispetto al valore vero. L'errore è probabilmente dovuto a un errore sistematico nella velocità di rotazione dello specchio. 
\end{itemize}
 
