\subsection{Spettri atomici }
Il  fenomeno  riguarda  l’interazione  tra  materia  allo  stato  gassoso  e  radiazione elettromagnetica. Quando un atomo viene investito da un fascio di fotoni accade che un elettrone, assorbendo un fotone, passi ad un livello energetico superiore, o stato eccitato. Questo stato è instabile e, dopo un tempo finito, l'elettrone tenderà a tornare allo stato di energia minima. La transizione da un livello energetico eccitato a uno stabile è accompagnata da emissione di radiazione. 
Si parla di assorbimento per la transizione di verso opposto.\\
%
Lo spettro di emissione o assorbimento di un materiale presenta in generale una moltitudine di informazioni circa la natura atomica, fisica e termodinamica dell'entità emittente. Alcuni tipici esempi sono temperatura, velocità, composizione chimica.
%
Per gas in condizioni di temperatura e pressione standard lo spettro spazia dalla radiazione infrarossa a quella ultravioletta in funzione della configurazione atomica.\\
%
Lo spettro di emissione nel visibile non è continuo. La presenza di righe spettrali di maggiore intensità nel caso degli spettri di emissione, o di righe scure in quelli di assorbimento, è indice del fatto che solo alcuni livelli energetici sono consentiti.
La distribuzione dei livelli energetici consentiti, attraverso la localizzazione delle lunghezze d'onda delle righe spettrali di un determinato elemento permette di caratterizzare la sua struttura atomica.
%
%
\subsection{Interferenza e diffrazione}
Quando un'onda attraversa un ostacolo opaco parte viene assorbita dalla superficie dell'ostacolo, producendo una deviazione dei raggi adiacenti. Tale fenomeno è detto diffrazione e può essere ben descritto tramite il principio di Huygens:\\
\textit{Nella propagazione di un'onda, ogni punto del fronte d'onda può essere considerato sorgente di onde secondarie, di modo che il fronte d'onda all'istante successivo risulti essere l'inviluppo delle onde secondarie che originano dal fronte d'onda precedente.}\\
Si hanno quindi due visioni perfettamente equivalenti per la descrizione dell'onda che propaga: come unico fronte d'onda in moto o come inviluppi successivi. Con questa descrizione, quando un'onda investe un ostacolo (come ad esempio una fenditura) i punti del fronte d'onda in diretto contatto con l'ostacolo vengono assorbiti mentre i rimanenti diventano sorgenti di onde secondarie. Il fronte d'onda oltre l'ostacolo diventa l'inviluppo dei fronti delle onde secondarie emesse dai punti non assorbiti dall'ostacolo.

Quando due onde si sovrappongono in una regione di spazio la perturbazione risultante è la somma delle due perturbazioni, secondo il Principio di Sovrapposizione. Nei punti in cui l'intensità risultante si annulla si parla di interferenza distruttiva, mentre nei punti in cui l'intensità risultante è massima si parla di interferenza distruttiva. 
Se si considerano onde piane sinusoidali di stessa lunghezza d'onda, del tipo:
    $$E_{1,2}(r,t) = E_{01,02}\cos(kr_{1,2}-\omega t +\phi_{1,2})$$
in un generico punto P, l'intensità dell'onda risultante sarà data da:
    $$ I = I_1 + I_2 + 2\sqrt{I_1I_2}\cos\delta $$
con $\delta = k(r_2-r_1) = k\Delta r$. Si vede che l'intensità è massima (interferenza costruttiva per $\cos\delta = 1$ e minima (interferenza distruttiva) per $\cos\delta = -1$, quindi:
    $$I_{max} \; \mathrm{per} \; k\Delta r = 2n\pi
    \quad\mathrm{e}\quad I_{min} \;\mathrm{per}\; k\Delta r = (2n+1)\pi $$
Se si considera un'onda composta da più lunghezze d'onda, poichè $ k = \lambda/2\pi $, si ha che lunghezze d'onda diverse hanno picchi di interferenza in punti diversi. Si ha così una scomposizione dello spettro dell'onda incidente e si possono ricavare le singole lughezze d'onda dalla condizione di interferenza per l'apparato sperimentale considerato.

\subsection{Rifrazione}
Quando un'onda elettromagnetica passa da un mezzo all'altro viene variata la velocità di propagazione a seconda delle proprietà elettromagnetiche ($\epsilon$ , $\mu$) del mezzo. Detto $n_k$ (indice di rifrazione) il rapporto tra la velocità della luce nel vuoto e nel mezzo k, le condizioni al contorno per il campo elettromagnetico sulla superficie di separazione tra i mezzi fanno si che l'onda subisca una deviazione tale da rispettare la legge di Snell:
    $$n_1\sin\theta_1 = n_2\sin\theta_2$$
Poichè l'indice di rifrazione del mezzo varia a seconda della lunghezza d'onda come:
    $$ n = A +  B/\lambda^2 $$
si ha che lunghezze d'onda differenti subiscono deviazioni differenti

\subsection{Fenomeno osservato e scopo dell'esperimento }
%
Il fenomeno osservato riguarda l'emissione di radiazione luminosa visibile generata da un gas rarefatto contenuto in un ampolla di vetro con due elettrodi connessi ad un'elevata differenza di potenziale.\\
%
L'emissione, una volta collimata e direzionata verso un dispositivo rifrangente, viene scomposta nelle lunghezze d'onda che la compongono.\\
%
L'oggetto della rilevazione sono perciò le righe spettrali e il relativo colore, la loro larghezza e distribuzione.\\
%
Su tale analisi influisce fortemente il mezzo rifrangente (analizzatore).\\
%
%
\paragraph{Scopo dell'esperimento }

\begin{itemize}
    
    \item misurazione delle righe spettrali di alcuni gas con l'obiettivo di identificare il materiale emittente.
    
    \item confronto di due mezzi rifrangenti, prisma e reticolo
    
\end{itemize}
%
%
%
\subsection{Descrizione dell'apparato }
%
\paragraph{Tubi di Plucker}{Un gas contenuto in un tubo lungo di vetro contente due elettrodi agli estremi viene sottoposto a una differenza di potenziale elevata. Gli elettroni del gas vengono quindi eccitati e, diseccitandosi, emettono luce.}
\paragraph{Spettroscopio}{Apparato costituito da una base con un nonio per la misura degli angoli, un telescopio che può essere allineato con la riga spettrale tramite un mirino posto sulla lente oculare.
La base era dotata di una rotella per affinare la misura, che purtroppo era rotta. Non è stato possibile sfruttare tutta la precisione del nonio in quanto, manualmente, l'errore variava al minuto d'arco, non al secondo.}
\paragraph{Prisma}{Prisma di vetro con un angolo al vertice (che è stato usato per la diffrazione della luce) di $62 \pm 1 °$}
\paragraph{Reticolo di diffrazione}{Essendo costituito da un film incollato su una lastra di vetro spessa, per evitare gli effetti di rifrazione
della luce nel vetro si è fatto in modo che la lastra di vetro fosse attraversata
dalla luce proveniente dal collimatore prima di essere diffratta dal reticolo. Nominalmente, la densità del reticolo è di 600 linee/mm che corrisponde a un passo d = 1667 nm}
