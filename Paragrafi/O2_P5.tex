\subsection{Interferenza da righello}
Quando il fascio laser incide con un angolo molto vicino al piano del righello, si osserva interferenza. Questa è dovuta al fatto che le tacche dei millimetri si comportano da reticolo e i punti nello spazio tra le tacche come sorgenti coerenti. Sebbene la larghezza delle tacche sia molto maggiore della lunghezza d'onda, siccome il fascio è molto inclinato, la lunghezza d'onda "vista" dal reticolo è comparabile con la dimensione delle fenditure.\\\\
%
La figura di interferenza è descritta dalla seguente formula:
%
    $$ d(\cos\theta_i - \cos\theta_r) =  N \lambda $$
%
dove $d$ è il passo righello (1 mm), $\theta_i$ l'angolo di incidenza del laser, $\theta_r$ angolo di riflessione del massimo di ordine $N$.\\
Detti $y_0$ la posizione sullo schermo del massimo di ordine zero rispetto al piano del righello, $y_N$ la posizione del massimo di ordine $N$ e $L$ la distanza dello schermo dal punto di incidenza del laser sul righello, gli angoli di incidenza e riflessione si trovano così:
$$ \theta_i = \alpha = \mathrm{arctg}(y_0 / L) \quad ; \quad \theta_{r_N} = \mathrm{arctg}( y_N / L ) $$
%
Tabella risultati ($L = 115.5 \pm 1$ cm):
\begin{table}[H]
    \centering
    \begin{tabular}{|c|c|c|c c|}
    \hline
         ordine max	& $\theta_i$ & $\theta_r$ & $\lambda$ & errore $\lambda$\\
            -	&	°	&	°	&	nm	&	nm	\\
            -	& ($\pm 0.09$ ) & ($\pm 0.09$ ) & -	& -	\\
    \hline
            1	&	3.42	&	3.99	&	640	&	12	\\
            -1	&	3.42	&	2.78	&	606	&	10	\\
            2	&	3.42	&	4.55	&	689	&	14	\\
            -2	&	3.42	&	1.88	&	619	&	9	\\
    \hline
    \end{tabular}
    %\caption{Caption}
    %\label{tab:my_label}
\end{table}
si ottiene:
    $$ \lambda = 629 \pm 5 \; nm $$
In perfetto accordo con il valore nominale $633$ nm.
