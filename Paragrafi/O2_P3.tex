\subsection{Indice di rifrazione dell'aria}
L'indice di rifrazione dell'aria $n$ ha un andamento lineare con la pressione $P$, secondo la relazione $n = mP+1$.\\
%
È stato ricavato il coefficiente $m$ misurando $\Delta k$ per diversi valori di $(P_f-P_i)$, con la formula
%
    $$ m = \frac{\Delta k  \lambda_0}{2d(P_f-P_i)} $$
%
I dati sperimentali sono:
%
\begin{table}[H]
    \centering
    \begin{tabular}{|c|c|}
    \hline
        $\Delta P$	&	$\Delta k$	\\
        $kPa$	&	$-$	\\
        ($\pm2$)	&	($\pm 1$)	\\
        \hline
        78	&	17	\\
        84	&	18	\\
        84	&	17	\\
        48	&	10	\\
    \hline
    \end{tabular}
\end{table}
%
Il valore ottenuto (mediante media pesata) è $ m = 2.3 \pm 0.2 \cdot10^{-9}$ $\mathrm{Pa}^{-1}$.\\\\
%
Noto $m$ e usando $\lambda_0$ ricavata dall'esperimento di Michelson, si può calcolare $n_{aria}$ utilizzando come valore di $P$ il valore della pressione atmosferica ($101.5 \mathrm{kPa}$).
    $$ n_{aria} = 1+mP_{atm} =  1.00023 \pm 0.00002 $$ 
%
%
Il valore vero di $n$ a pressione atmosferica è $1.00029$, dunque a ritroso, il valore teorico di $m$ è $ 2.9 \cdot10^{-9}$ $\mathrm{Pa}^{-1}$. L'errore rispetto al valore teorico è del $21\%$.