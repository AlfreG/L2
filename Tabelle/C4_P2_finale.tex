\begin{table}[H]
\begin{center}
\begin{tabular}{|c|c|c|c|c|c|}

\hline
\multicolumn{ 1}{|c|}{Par.} & \multicolumn{ 1}{c|}{U.tà} & \multicolumn{ 3}{c|}{Funzioni di trasferimento RLC – AC} & \multicolumn{ 1}{c|}{RLC – DC} \\ \cline{3-5}

\multicolumn{ 1}{|c|}{} & \multicolumn{ 1}{c|}{} & Res & Cond & Ind & \multicolumn{ 1}{c|}{} \\ \hline

$r$ & $\Omega$ & $104\pm8\%$ &  &  & $100\pm1\%$ \\ 
$L$ & $H$ & $0.077\pm2\%$ &  &  & $0.066\pm2\%$ \\ 
$C$ & $nF$ & $41\pm4\%$ &  &  & $46\pm3\%$ \\ \hline
$R$ & $\Omega$ & \multicolumn{ 3}{c|}{$1000\pm1\%$} & $200\pm1\%$ \\ 
\hline

\end{tabular}
\end{center}
\caption{
Stime di $r$, $L$ e $C$.
$r$: resistenza parassita del circuito, comprensiva della resistenza del generatore e quella interna dell'induttore.
$L$: induttanza.
$C$: capacità.
$R$: resistenza del resistore non è stata oggetto di stima in quanto misurata direttamente, e riportata qui per completezza.
Nell'ultima colonna sono riportati i valori stimati nella Parte 1 in corrente continua.
}
\label{C4_P2_finale}
\end{table}
